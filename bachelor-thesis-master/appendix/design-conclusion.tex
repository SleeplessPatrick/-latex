\newpage
\mbox{}
\newpage
\renewcommand{\baselinestretch}{1.5}
\fontsize{12pt}{13pt}\selectfont
\phantomsection
\chapter*{毕业设计小结}
\addcontentsline{toc}{chapter}{\fHei 毕业设计小结}
%随着毕业日期的逼近,我们的毕业设计也即将划上句号。毕业设计是我们学业生涯的最后一个环节,不仅是对所学基础知识和专业知识的一种综合应用,更是对我们所学知识的一种检测与丰富,是一种综合的再学习、再提高的过程,这一过程对我们的学习能力、独立思考及工作能力也是一个培养。

%在没有做毕业设计以前觉得毕业设计只是对这几年来所学知识的单纯总结,但是通过这次做毕业设计发现自己的看法有点太片面。毕业设计不仅是对前面所学知识的一种检验,而且也是对自己能力的一种提高。通过这次毕业设计,我才明白学习是一个长期积累的过程,在以后的工作、生活中都应该不断的学习,努力提高自己知识和综合素质。
。
%我们设计毕业论文就是运用已有的专业基础知识,独立进行科学研究活动,分析和解决一个理论问题或实际问题,把知识转化为能力的实际训练。毕业设计是对我们的知识和相关能力进行一次全面的考核,是对我们进行科学研究基本功的训练,培养我们综合运用所学知识独立地分析问题和解决问题的能力,为以后撰写专业学术论文打下良好的基础。

%我认为,毕业设计也是对在校大学生最后一次知识的全面检验,是对学生基本知识、基本理论和基本技能掌握与提高程度的一次总测试。毕业论文不是单一地对学生进行某一学科已学知识的考核,而是着重考查学生运用所学知识对某一问题进行探讨和研究的能力。

%毕业论文的过程是训练我们独立地进行科学研究的过程。撰写毕业论文是学习怎么进行科学研究的一个极好的机会,有指导教师的指导与传授,可以减少摸索中的一些失误,少走弯路,而且直接参与和亲身体验了科学研究工作的全过程及其各环节,是一次系统的、全面的实践机会。撰写毕业论文的过程,同时也是专业知识的学习过程,而且是更生动、更切实、更深入的专业知识的学习。

%毕业设计论文是结合科研课题,把学过的专业知识运用于实际,在理论和实际结合过程中进一步消化、加深和巩固所学的专业知识,并把所学的专业知识转化为分析和解决问题的能力。同时,在搜集材料、调查研究、接触实际的过程中,既可以印证学过的书本知识,又可以学到许多课堂和书本里学不到的活生生的新知识。此外,学生在毕业论文写作过程中,对所学专业的某一侧面和专题作了较为深入的研究,会培养学习的志趣,这对于我们今后确定具体的专业方向,增强攀登某一领域科学高峰的信心大有裨益。所以毕业设计的研究对我们来说,意义非凡。
%在此要感谢我的指导老师布树辉老师对我悉心的指导,感谢老师给我的帮助。在设计过程中,我通过查阅大量有关资料,与同学交流经验和自学,并向老师请教等方式,使自己学到了不少知识,也经历了不少艰辛,但收获同样巨大。

%在整个设计中我懂得了许多东西,也培养了我独立工作的能力,相信会对今后的学习工作生活有非常重要的影响。毕业设计的研究期间,我大大提高了动手能力,使我充分体会到了在创造过程中探索的艰难和成功时的喜悦。在此,我向帮助我的老师和同学们表示衷心的感谢!



\clearpage
\endinput