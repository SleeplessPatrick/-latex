\chapter{绪论}\label{preface}

\section{研究背景及意义}
水动力噪声是水下高速航行器的主要噪声源。航行器壁面的边界层转捩会使其边界层中层流演变成湍流,在湍流边界层中随机的速度扰动会产生脉动压力,随着这种随机的脉动压力产生的噪声统称为水动力噪声,其强度随航速的增加而迅速增加,辐射声功率正比于航速的5$\sim$7次方 \ucite{俞孟萨2007国外舰船水动力噪声研究进展概述},若不发生空化,边界层噪声是水动力噪声的主要噪声源,其主要由边界层转捩区噪声和湍流边界层噪声构成。既然产生了噪声就有必要对其加以控制,降噪有诸多好处:提高声呐探测精度,进而提高鱼雷命中率和潜艇生存力;减小对内部系统工作的影响;减小阻力,进而降低能耗提高效率,增大航程。水动力噪声的降噪主要通过推迟转捩实现。但水下高速航行器在海水中保持长时间的工作状态航行时,由于电化学腐蚀和海洋寄生物侵蚀等因素往往会形成微粗糙表面,这是影响边界层转捩的重要因素之一,在高速流动条件下,随着粗糙元高度增加,转捩区域明显扩大并有提前的趋势 \ucite{赵云飞2015粗糙物面引起的超声速边界层转捩现象研究}。针对该特征,拟通过毕业设计,构建能够精确预测粗糙度对水下高速航行器表面的转捩-湍流特性影响的预测方法,分析壁面粗糙度对转捩-湍流特性的影响,在本文工作的基础上能够进一步合理优化航行器外形并对转捩致湍流脉动噪声进行控制。


\section{湍流模拟方法概述}
当前主流的湍流的数值模拟方法主要分为三种:直接数值模拟(Direct Numerical Simulation,DNS),大涡模拟(Large Eddy Simulation,LES)及雷诺平均方法(Raynolds averaged Navier-Stocks,RANS)。DNS方法直接求解湍流中所有的涡结构,其最小模拟尺度小于耗散尺度,结果中包含丰富的脉动信息,但随着求解目标的湍流雷诺数增大,最小湍流尺度相对于最大湍流尺度减小,DNS计算所需网格量就会增大,进而使得计算成本大大增加,在工程上仍不具有可实现性。RANS方法对湍流中不同尺度的涡结构进行模化,平均求解所有涡结构信息,对网格分辨率要求较低,忽略了湍流流动的细节,计算精度较低,具有较好的鲁棒性和经济型,可以模拟大部分的简单流动现象,因此在工程上得到了广泛应用。目前广泛使用的湍流模型大多基于RANS方法。Spalart等人 \ucite{spalart1992one}从经验公式和量纲分析法出发,提出了Spalart-Allmaras一方程模型(SA模型),该模型对网格要求不是很高,能较好地处理边界层中粘性区域,但其精度不高,且欠缺对复杂流动进行模拟的能力。在RANS方法发展过程中第一个实用化的两方程模型是Jones等人\ucite{jones1973calculation}提出的$\textkappa-\varepsilon$模型,但其边界条件处理困难,鲁棒性差,不能很好的模拟逆压梯度、流动分离。Wilcox \ucite{wilcox1988reassessment}提出的$\textkappa-\textomega$模型是应用最为广泛的二方程湍流模型之一,其比$\kappa-\varepsilon$模型的鲁棒性更好,便于应用在复杂外形流动中,且能很好的模拟逆压梯度、流动分离,但存在对自由来流条件比较敏感的缺陷。Menter\ucite{menter1993zonal}提出的$\textkappa-\textomega$剪切应力输运模型(Shear Stress Transport,SST)在近壁面区采用$\textkappa-\textomega$模型,在边界层外缘和自由剪切区使用$\textkappa-\varepsilon$模型,并通过Bradshaw假设(边界层中剪切应力正比于湍动能)引入了雷诺剪切应力输运的影响。其吸收了$\textkappa-\varepsilon$模型、$\textkappa-\textomega$模型、JK模型的优点,该模型能很好预测逆压梯度和流动分离,以其适用性广、精度高、鲁棒性好的优势成为了应用最为广泛的湍流模型之一。LES方法则介于DNS方法和RANS方法之间,由Smagorinsky \ucite{smagorinsky1963general}最早提出:将小尺度的涡模化,降低了在高雷诺数流动时使用DNS方法计算带来的高计算成本;对大尺度的涡直接求解,捕捉到使用RANS方法时难以预测的涡分离流动,弥补了使用RANS方法在计算复杂流动时精度不高的问题。但LES方法在近壁处存在缺陷:在近壁面的湍流流动中,有着大量脉动尺度正比于壁面距离的小尺度的湍流脉动,随着雷诺数增大,湍流尺度变小,为了求解壁面域需要设置大量网格点,进而带来巨大计算量。
\section{转捩预测方法概述}
在转捩预测方法上,基于RANS的转捩模型具有较好的精度、鲁棒性和经济性,因此被大量用于工程中。基于RANS方法的转捩模型是主要的转捩预测方法,其虽然无法捕捉到足够的脉动信息,但计算量较小且预测精度较高,所需变量能当地化求解,兼有可以与现代大型CFD求解器兼容的优点。在转捩模型研究的最初期,Wilcox 等人 \ucite{wilcox1998turbulence, wilcox1993turbulence}发现低雷诺数湍流模型具有预测转捩的能力,但既没有考虑转捩的间歇现象,也没有考虑层流动能;Dhawan和 Narasimha \ucite{narasimha1985laminar, dhawan1958some}最早根据平板边界层实验给出了间歇因子 的经验关系式。而后众多学者们基于拟合公式构造出$\gamma$输运方程,比较有代表性的有Steelant和Dick  \ucite{steelant2001modeling}基于Dhawan和Narasimha的经验公式 \ucite{narasimha1985laminar, dhawan1958some}构造出的$\gamma$输运方程。但间歇因子只描述转捩过程,为了预测转捩还需要转捩起始位置判据,这一判据通常基于动量厚度雷诺数$Re_\theta$发展而来,在层流流场中一旦超过由经验确定的临界值,即认为发生转捩。Menter和Langtry \ucite{langtry2006correlation, menter2005transition, langtry2009correlation}基于大量实验数据拟合了转捩动量厚度雷诺数的经验公式,并使用涡量雷诺数$Re_\upsilon$替代动量厚度雷诺数$Re_\theta$进行当地化求解,提出了$\gamma-\overline{Re_\theta}$转捩模型。该模型所有变量均可当地化求解,与现代CFD技术兼容,具有大规模并行计算的能力。研究人员们针对基于$\textkappa-\omega~SST$湍流模型的$\gamma-\overline{Re_\theta}$转捩模型在水下航行器表面的转捩预测方面也进行了诸多探索。Rahman等人 \ucite{rahman2022capturing}对$SST~\gamma-\overline{Re_\theta}$转捩模型进行了修正,修改了混合函数$F_2$,并引入了应力-强度参数作为涡流与层流粘度比$R_T$的函数。结果表明该模型能很好预测低雷诺数水翼的边界层转捩现象。Delafin等人 \ucite{delafin2014effect}研究了层流分离泡诱导转捩对水翼升力、阻力和力矩系数的影响,结果表明$SST~\gamma-\overline{Re_\theta}$转捩模型在中等雷诺数条件下能很好地捕捉水翼表面的转捩现象。Ye等人 \ucite{ye2021improvement}通过对$SST~\gamma-\overline{Re_\theta}$转捩模型中的转捩起始雷诺数$Re_{{\theta}t}$进行修正,提高了该模型对弯曲水翼边界层转捩的预测精度
\section{粗糙表面模拟方法概述}
众多研究人员们针对壁面粗糙度问题对湍流模型进行了修正。考虑壁面粗糙度的湍流模型修正重要方法之一是等效沙砾法,在工程中有着广泛的应用。Nikuradse在其1933年的粗糙管道实验 \ucite{nikuradse1950laws}中首次提出等效沙砾法的概念,即通过沙砾高度描述表面粗糙度,并将沙砾高度作为新参数导入模型,进而增强壁面湍流。在此基础上,研究人员们针对各个湍流模型提出了不同的修正方法。Durbin等人\ucite{2001Rough}通过引入流体动力学粗糙度长度,并修改边界条件以在完全粗糙和完全光滑之间进行插值,为两层 模型建立了一个粗糙壁模型。Aupoix等人 \ucite{aupoix2003extensions}将一方程Spalart-Allmaras(SA)湍流模型用于粗糙壁面,根据壁面涡粘度和粗糙高度之间关系进行方程的修正。Knoop等人 \ucite{knopp2009new}提出了可用于$\textkappa-\omega~SST$模型的粗糙壁面公式,对壁面湍动能和比耗散率进行修正。建立在考虑壁面粗糙度对湍流模型进行修正的基础上,研究人员们进一步考虑表面粗糙度对边界层转捩模型进行修正: Ge等人 \ucite{ge2015intermittency}使用了有效原点对光滑表面转捩模型的壁距离进行修改,该原点取决于等效沙砾粗糙度。该模型能够根据转捩位置和粗糙度高度、雷诺数和入口湍流强度预测转捩开始。Dassler \ucite{dassler2010transport}等人引入了新的变量粗糙增长因子$A_r$到$\gamma-\overline{Re_\theta}$转捩模型中,$A_r$的值取决于壁面上的等效沙砾粗糙度高度,会通过附加的转运方程在流场中对流并耗散。通过$A_r$输运方程和$Re_{{\theta}t}$输运方程的相互作用,使得$Re_{{\theta}t}$减小,转捩位置提前。Langel等人 \ucite{langel2015further}通过加入粗糙增长因子$A_r$来扩充Langtry-Menter转捩模型,该额外的变量和输运方程允许将表面粗糙度的非当地影响传递到粗糙段下游,该模型还引入了局部压力梯度的影响和粗糙度密度变化进行修正。Yang等人 \ucite{yang2019distributed}提出了四方程$\textkappa-\omega-\gamma-A_r$转捩模型,从指定壁面上的边界条件中的对流和扩散获得$A_r$的空间分布,$A_r$的作用是放大有效长度尺度,这导致放大了第一模态时间尺度,从而导致更早的转捩,同时提出了对第一模态时间尺度的修改,以明确反应湍流强度(Tu)的影响。Liu等人 \ucite{liu2020predicting}基于层流动能转捩模型(${{\textkappa}_T}-{{\textkappa}_L}-{\omega}模型$)建立了四方程层流动能转捩模型,将粗糙度放大因子的附加输运方程与层流动能转捩模型相结合,再通过粗糙度放大因子修改有效长度尺度,以放大第一模态和第二模态的特征时间尺度。此外为了模拟全湍流区的粗糙度效应,修正了特定湍流耗散率的壁面边界条件。Zhang等人 \ucite{zhang2022modeling}提出了一种结合粗糙度放大因子输运方程的间歇转捩模型,以预测粗糙表面上转捩的开始。将粗糙度修正应用于壁边界条件,对具有不同粗糙表面的平板和涡轮叶片的预测结果与实验数据吻合良好。
\section{本文主要工作安排} 
本毕业设计研究的目的是构建能够精确预测粗糙度对水下高速航行器表面的转捩-湍流特性影响的预测方法,研究壁面粗糙度对转捩-湍流特性的影响。

本文工作安排如下:

第一章~引言。~说明研究背景,研究目标和研究意义,并介绍国内外的研究现状。

第二章~转捩-湍流预测方法。~介绍$SST~\gamma-\overline{Re_\theta}$转捩模型,并通过平板算例、S809风力机翼型算例和NLF0416翼型算例来验证使用该转捩模型能够准确预测转捩现象。

第三章~考虑粗糙壁面的转捩预测方法。~首先介绍耦合$A_r$输运方程的$SST~\gamma-\overline{Re_\theta}$转捩模型,第一步是介绍粗糙因子$A_r$输运方程,接下来说明如何将粗糙度影响耦合在$\gamma-\overline{Re_\theta}$转捩模型上。最后通过粗糙平板算例、粗糙NACA0012翼型和粗糙S814翼型来验证使用耦合$A_r$输运方程的$SST~\gamma-\overline{Re_\theta}$转捩模型能够准确预测粗糙转捩现象。

第四章~水下航行器粗糙转捩特性分析。~本次毕业设计的目标是构建能够精确预测粗糙度对水下高速航行器表面的转捩-湍流特性影响的预测方法,因此基于以上章节的粗糙转捩模型,需要进行对水下航行器粗糙转捩特性进行分析,针对suboff潜艇和鱼雷在不同的实验状态下进行分析。

第五章~总结与展望。~针对本文工作进行总结,说明本文工作中需要进一步改善的问题,并对未来可以继续深入研究的方向进行展望。

\clearpage
\endinput