%\newpage
%\mbox{}
%\newpage
\chapter{转捩-湍流预测方法}\label{introduction}
\section{$SST~\gamma-\overline{Re_\theta}$转捩模型概述}


 
%\begin{equation}
%\mequlist{x_k=f(x_{k-1},u_k,w_k) \\ z_{k,j}=h(y_j,x_k,w_{k,j})}
%\label{eqn_slam_theory}
%\end{equation}

\section{转捩模型算例验证}
%闭环检测是视觉 SLAM 的重要组成部分,经过前端视觉里程计计算相邻帧数据、后端进行状态最优估计之后,所得到的数据直接用于构造地图会有明显误差,因为视觉里程计对摄像机的初始运动估计存在误差并随着时间的进行产生累积误差,无法构建全局一致的轨迹和地图。如图 \ref{fig_without_loopclose}和图 \ref{fig_with_loopclose}所示,闭环检测算法会对全局地图增加一个约束关系,使得机器人生成的地图更加准确。
\subsection{无压力梯度平板}

\subsection{有压力梯度平板}

\subsection{S809风力机}

\subsection{Nlf(1)-0416}

%\begin{figure}[htbp]
%	\centering
%	\includegraphics[scale=0.6]{figures/chap2-without closed-loop detection.png}
%	\caption{未经闭环检测}
%	\label{fig_without_loopclose}
%\end{figure}

%\begin{figure}[htbp]
	%\centering
	%\includegraphics[scale=0.6]{figures/chap2-with closed-loop detection.png}
	%\caption{经过闭环检测修正后}
	%\label{fig_with_loopclose}
%\end{figure}

%\subsection{闭环检测过程}
%闭环检测的核心问题主要有图像特征的提取与描述、视觉场景建模、图像检索和图像间相似性度量四个部分,其流程图如下所示:

%\begin{figure}[ht]
%	\centering
%	\includegraphics[scale=0.6]{figures/chap2- closed-loop detection.png}
%	\caption{闭环检测流程图}
%\end{figure}




 
 \section{本章小结}
%本章阐述了视觉 SLAM 闭环检测的相关原理和方法。首先,叙述了视觉 SLAM 的基础理论,并着重介绍了视觉 SLAM 闭环检测的实现方法和组成模块。其次,对基于场景外观的闭环检测方法的场景图像描述、闭环决策模型等模块的传统方法存在的不足进行了探讨,为后续深入研究做好铺垫。

\clearpage
\endinput
