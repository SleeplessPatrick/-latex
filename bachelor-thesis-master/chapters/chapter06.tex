\chapter{全文总结}
视觉 SLAM 作为移动机器人各种应用的基础以及自动驾驶、VR、AR 等应用的重要组成部分受到了极大的关注,是移动机器人视觉导航中的核心问题之一。闭环检测作为视觉 SLAM 的重要环节,是机器人判断自己当前位置是否位于已访问过的环境区域,成功检测出闭环,可以显著地减小前端视觉里程计的累积误差,并以此作为地图是否需要更新校正的依据,决定着机器人定位的精度与地图构建的一致性,然而由于传统算法的局限性,目前的闭环检测算法仅仅能有效的应对小范围的静态室内环境。鉴于现实场景的复杂性,本文针对闭环检测过程中的场景图像描述、闭环决策模型等模块进行了深入系统的研究,结合深度学习在图像分类与检索任务中的优异性与传统方法的实用性,提出了基于融合 CNN 与 VLAD 特征的图像描述。

本文完成的主要工作总结如下:

1.查阅大量国内外文献资料,分析了移动机器人视觉导航中的视觉 SLAM 问题,概述了视觉 SLAM 和闭环检测的研究现状,重点阐述了现有闭环检测方法存在的问题。

2.通过对视觉 SLAM 的基础理论的分析以及对视觉 SLAM 闭环检测不同实现方法的总结,对视觉 SLAM 闭环检测的组成模块进行了提炼,并探究了不同模块的现有实现方法及其不足之处。

3.比较了 SIFT、SURF 和 ORB 这三种特征点算法的优缺点,针对视觉 SLAM 闭环检测的场景图像描述模块,深入分析了基于传统特征与基于深度学习的图像描述及其存在的问题,然后结合神经网络与 VLAD特征的特性,提出基于融合 CNN 与 VLAD 特征的图像描述方法。

4.
详细介绍了 VGG16-NetVLAD 卷积神经网络模型的框架,然后介绍了模型的训练参数的设置,然后给出了实验用的闭环检测方法,用余弦相似度来计算两特征向量之间的相似性,最后在标准的闭环检测数据集 New College 和City Center上进行测试,分别和几种传统基于人工设计特征的方法(BoVW)以及其他几种深度学习模型的方法进行了对比验证,实验结果表明本文采用的VGG16-NetVLAD 卷积神经网络模型在闭环检测的 PR 性能和特征提取时间性能上具有比较好的优势,为视觉 SLAM 闭环检测提供了一种新方法。

本文只对一些方面做出了研究,仍有许多问题有待于进一步研究与完善,具体包括以下几个方面: 

1.对于主流的传统人工特征方法在特征提取与描述子计算方面要尝试使用更新更快的算法进一步提高闭环检测算法的效率。 

2.要设计更加精简、快速而又对于闭环检测效果比较好的深度模型,本文用的模型虽然效果比较好,但是训练时间长,提取特征文件所占资源也较大,因此更高配置的计算机资源也是必要。

3.本文采用对比验证试验还略显不够,没有考虑光照等条件变化对各种算法能否成功检测出闭环的影响的对比,后续尝试增加对该方面试验进行验证。同时目前算法在复杂光照条件下进行的闭环检测准确度较低,解决这一问题可帮助视觉 SLAM 技术得到更加广泛的应用,提升视觉 SLAM 技术的普及程度。

4. 深度学习方法与视觉 SLAM 技术结合的其他场景应需继续研究,传统视觉 SLAM 只能生成传统的地图,而与深度学习结合,可以生成语义地图,使得机器人完成更复杂的功能来满足人们的需求。


\endinput