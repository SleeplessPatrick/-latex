\renewcommand{\baselinestretch}{1.5}
\fontsize{12pt}{13pt}\selectfont

\chapter*{摘~~~~要}
\markboth{中~文~摘~要}{中~文~摘~要}

%随着近年来视觉同时定位与地图构建(Visual Simultaneous Localization and Mapping,VSLAM)的快速发展,闭环检测已经成为了移动机器人导航领域的关键问题和研究热点,。闭环检测本质上来说很像一个分类问题,鉴于深度卷积神经网络在图像分类和图像检索任务方面的优越表现,本文提出一种基于卷积神经网络和VLAD的网络模型结构,将该模型作为一种新的视觉SLAM闭环检测方法。

%本文首先对视觉SLAM的研究现状加以介绍,包括基于滤波器和基于图优化的视觉SLAM研究进展。接着详细介绍了作为视觉SLAM重要环节的闭环检测的研究现状,包括基于人工设计特征以及基于深度学习的闭环检测的国内外研究情况。
 
%其次,本文分析了视觉SLAM的基础理论,详细介绍了VSLAM的整体流程。总结了视觉 SLAM 中闭环检测的流程和不同实现方法,对视觉 SLAM 闭环检测的组成模块进行了提炼,并探究了不同模块的现有实现方法及其不足之处。

%接着本文对特征点检测算法进行了研究,分析对比SIFT、SURF和ORB特征描述子方法。针对视觉 SLAM 闭环检测的场景图像描述模块,深入分析了基于传统特征与基于深度学习的图像描述及其存在的问题,然后结合神经网络与 VLAD特征的特性,提出基于融合VGG16网络与NetVLAD特征提取的网络模型框架,介绍了模型的弱监督训练过程及损失函数的计算,给出了实验用的闭环检测方法:用余弦相似度来计算模型提取到的特征向量之间的相似性并构建相似性矩阵,通过比较阈值判断是否构成闭环。

%最后,本文将基于 VGG16-NetVLAD网络模型的闭环检测方法与传统的词袋模型闭环检测方法以及单纯基于VGG16卷积神经网络的闭环检测方法在数据集 New College和City Center上进行了对比实验,对实验结果加以分析,实验结果表明基于VGG16-NetVLAD网络模型在闭环检测上的 PR 性能和特征提取时间性能上具有比较好的表现。

\vspace{-10pt}



\vspace{1em}
\noindent {\fHei 关键词:} \quad %卷积神经网络,闭环检测,视觉同步定位与地图构建,特征提取

\clearpage
\endinput
